\documentclass[10pt,preprint]{sigplanconf}

\usepackage{amsmath}
\usepackage{times}
\usepackage{comment}
\usepackage{amssymb}
\usepackage{amsmath}
\usepackage{ifthen}
\usepackage{multirow}
\usepackage{xspace}
\usepackage[pdftex]{graphicx}
\usepackage{url}
\usepackage{color}
\urlstyle{sf}
\usepackage{makeidx}
%\usepackage{cite}
\usepackage{url}
\usepackage{rotating}
\usepackage{multirow}

\newboolean{hidecomments}
\setboolean{hidecomments}{false}
\ifthenelse{\boolean{hidecomments}}
{\newcommand{\nb}[2]{}}
{\newcommand{\nb}[2]{
    \fbox{\bfseries\sffamily\scriptsize#1}
    {\sf\small$\blacktriangleright$
      {#2} $\blacktriangleleft$}}}
\newcommand\John[1]{\nb{John}{#1}}
%\newcommand\Danny[1]{\textbf{Danny: {#1}}} 
\newcommand\Danny[1]{\nb{Danny}{#1}}
\newcommand\todo[1]{\fbox{\bfseries\sffamily\scriptsize TODO: #1}}
\newcommand\tool{{\smaller\textsc{Concurrencer}}\xspace}
\newcommand{\Comment}[1]{}
\newenvironment{CodeOut}{\begin{scriptsize}}{\end{scriptsize}}
\newcommand{\code}[1]{\begin{small}\texttt{#1}\end{small}}
\newcommand{\myParagraph}[1]{\textbf{#1}}
\newcommand{\HalfWidth}{.45\columnwidth}
\newcommand{\FullWidth}{.95\columnwidth}
\newcommand{\MaxWidth}{\columnwidth}

%\hyphenation{Concurrency-Re-factor-er,re-fac-to-ring}

\begin{document}

% \conferenceinfo{Second Workshop on Refactoring Tools}{Oct. 19-23, Nashville} 
% \copyrightyear{2008} 
% \copyrightdata{[to be supplied]} 

%\titlebanner{banner above paper title}        % These are ignored unless
%\preprintfooter{short description of paper}   % 'preprint' option specified.

\title{Refactoring Code to Use Concurrent Library Utilities}
%\subtitle{- draft version 0.8 - }

\authorinfo{Danny Dig\and John Marrero\and Michael D. Ernst}
           {Massachusetts Institute of Technology}
           {dannydig,marrero,mernst@csail.mit.edu}

\maketitle
\thispagestyle{empty}

\begin{abstract}
Parallelizing existing sequential programs to run efficiently on multicores is
not trivial. The programmer needs to make the program \emph{thread-safe} (i.e.,
the program runs correctly when executed from several threads) and also \emph{scalable} (i.e., the performance of the program
improves when adding more parallel resources).

The Java class library has been extended with a package
\code{java.util.concurrent (j.u.c.)} which provides thread-safe and scalable
collections and utility classes. However, manual refactoring to such
utilities is tedious and error-prone. This paper presents \tool, our extension
to Eclipse's refactoring engine. \tool enables a programmer to refactor to 
two \code{j.u.c.} utilities, \code{AtomicInteger} and \code{ConcurrentHashMap}.
Preliminary experience with \tool shows that
\tool refactors code effectively: \tool correctly identifies and applies more
transformations than some open-source developers.

\end{abstract}

%\category{CR-number}{subcategory}{third-level}

% \terms
% Refactoring, concurrency, concurrent library, Java
% 
% \keywords
% keyword1, keyword2

\section{Introduction}

% The problem: multicores are here
% For several decades, the computing hardware industry has kept up with Moore's
% Law, effectively doubling the number of transistors per chip every 18 months.
% This translated in roughly doubling the speed of programs that ran
% on the newer machines, an effect dubbed as ``the free performance lunch''.
% Although the hardware industry keeps up with Moore's Law, from now it delivers
% the transistors into multicore chips that run in parallel. This paradigm shift
% demands that programmers find and exploit parallelism in their programs, if they
% want to reap the same performance benefits as in the past.

The computing hardware industry has shifted to multicore processors. This
demands that programmers find and exploit parallelism in their programs, if
they want to reap the performance benefits.


% The challenge: parallelize sequential applications
% In the multicore era, a major programming task is to retrofit parallelism into
% the existing sequential programs. It is arguably easier to design 
% the programs with concurrency in mind than to retrofit it
% later~\cite{Lea:CPJ99, Goetz:JCP06}. However, most of the desktop programs were
% not designed to be concurrent, thus programmers have to refactor existing
% sequential programs for concurrency.

% Concrete challenge: thread-safety and scalability
The dominant paradigm for concurrency in desktop programs is shared-memory,
thread-based. However, this paradigm increases the risk for deadlocks and
data-races, commonly known as \emph{thread-safety} concerns. In addition, the
programmer needs to consider \emph{scalability} concerns as well: will the
parallelized program continue to run faster when adding more parallel 
resources? 

% Java.util.concurrent class library
To meet the needs of programmers with respect to thread-safety and scalability,
the Java standard library has been extended with a package,
\code{java.util.concurrent} (from here on referred as \code{j.u.c.}), containing
several utility classes for dealing with concurrency. Among others,
\code{j.u.c.} contains a set of \code{Atomic} classes which offer thread-safe,
lock-free programming over single variables, and several thread-safe abstract
data types (e.g., ConcurrentHashMap) optimized for scalability. However,
manually refactoring a program to use \code{j.u.c.} utilities is
\emph{tedious} and \emph{error-prone}.

% This paper presents \tool and two automated refactorings:

This paper presents \tool, our extension to Eclipse's refactoring engine. \tool
enables Java programmers to quickly and safely refactor their sequential
programs to use \code{j.u.c.} utilities. In this paper we present two 
refactorings: (i) \emph{ConvertToAtomicInteger} and (ii) \emph{ConvertToConcurrentHashMap}.
Our previous study~\cite{Dig'08:studyOfConcurrentTransformations} of five
open-source projects that were manually parallelized by their developers shows
that these two refactorings were among some of the most commonly used in
practice.

% ConvertToAtomicInteger
First refactoring, \emph{ConvertToAtomicInteger}, enables a programmer to
convert an \code{int} field to an \code{AtomicInteger}. \code{AtomicInteger} is
a lock-free utility class which encapsulates an \code{int} value. Our
refactoring replaces field updates with calls to \code{AtomicInteger}'s APIs.

% This refactoring changes the declaration
% type of the field and it replaces all field updates with calls to the APIs
% provided by \code{AtomicInteger}. For example, a common update operation on
% an \code{int} field is to increment its current value with a given delta. To
% make this update thread-safe, a programmer needs to encapsulate within a
% \emph{synchronized block} three operations: (i) read the current
% value, (ii) add delta, and (iii) update the field. Had the programmer forget to
% include all three operations, a data-race might appear. In addition, due to the
% program having to frequently acquire and release the lock, the program does not
% scale due to lock-contention. \tool finds such read/add/update code snapshots
% and replaces them with a call to \code{AtomicInteger}'s \code{getAndAdd()}
% which \emph{atomically} executes the update without using any locks. 
 


% ConvertToConcurrentHashMap 
Second refactoring, \emph{ConvertToConcurrentHashMap}, enables a programmer to
convert an \code{HashMap} field to \\
\code{ConcurrentHashMap}.
\code{ConcurrentHashMap} is a thread-safe, highly scalable implementation for
hash maps. Our refactoring replaces map updates with calls to the APIs provided
by \code{ConcurrentHashMap}.

% This utility class
% enables all readers to access the map concurrently, without blocking, and
% allows a number of writers to execute updates concurrently. The refactoring
% changes the declaration type of the field and it replaces map updates with 
% calls to the APIs provided by \code{ConcurrentHashMap}. For example, a common 
% update operation is (i) check if a map contains a key-value pair, and if it is
% not present, then (ii) place the pair in the map. Traditionally, a programmer
% needs to encapsulate these two operations within a synchronized block. Since all
% accesses to a map have to aquire the map's lock, this can severely degrade the
% map's performance under heavy lock-contention. \tool finds such an updating operation and it replaces it with a call to
% \code{ConcurrentHashMap}'s \code{putIfAbsent} which \emph{atomically} executes
% the update without blocking (assuming that no other writers write in the same
% hash bucket).

% \tool implemented as an Eclipse plugin

% Preliminary experience with the two refactorings
To get some preliminary experience with \tool, we studied
the code that open-source developers manually refactored to make use
of \code{AtomicInteger} or \code{ConcurrentHashMap}. We also used \tool to
refactor the same code and then we compared the manually vs. automatically
refactored output. \tool applied all the
transformations that the developers applied. Even more, \tool correctly
identified and applied some transformations that the open-source developers
omitted. This experience shows that \tool is effective.

\tool can be downloaded from the webpage: \\
\code{http://refactoring.info/tools/\tool}

%--------------------------------------------
\section{Convert to Atomic Integer}

\begin{figure*}[t]
\begin{CodeOut}
\begin{tabular}{@{}l|l@{}}
\begin{minipage}[t]{\FullWidth}
\begin{verbatim}
// before refactoring
public class Counter {

 private int value = 0;
        
 public int getCounter() {
  return value;
 }
        
 public void setCounter(int counter) {
  value = counter;
 }
        
 public int inc() {
  return ++value;
 }
}
\end{verbatim}
\end{minipage}
&
\begin{minipage}[t]{\FullWidth}
\begin{verbatim}
// after refactoring
public class Counter {
 
 private AtomicInteger value = new AtomicInteger(0);
        
 public int getCounter() {
  return value.get();
 }
        
 public void setCounter(int counter) {
  value.set(counter);
 }
        
 public int inc() {
  return value.incrementAndGet();
 }
}
\end{verbatim}
\end{minipage}
\end{tabular}
\end{CodeOut}
\caption{Using \tool to refactor an \texttt{int} field to
\texttt{AtomicInteger} in some real Apache Tomcat code. Left/right shows code
before/after refactoring.}
\label{fig:CounterExample}
\end{figure*}

\subsection{AtomicInteger in Java}
Starting with \code{JDK 1.5}, the Java class library
offers a package \code{java.util.concurrent.atomic} that
supports \emph{lock-free} programming on \emph{single} variables. 

The package contains wrapper classes over primitive variables, for example,
an \code{AtomicInteger} wraps an \code{int} value. The main advantage
is that update operations execute atomically, without blocking. 
%For example,
% \code{AtomicInteger} provides an API method \code{addAndGet(int delta)} that
% atomically adds the given delta to the current value. 
Internally, the \code{AtomicInteger} employs efficient machine-level
atomic instructions like \emph{Compare-and-Swap} that are available on 
contemporary processors. Using the \code{AtomicInteger}, the programmer
gets both \emph{thread-safety} (built-in the \code{Atomic} classes) and
\emph{scalability} (the lock-free updates reduce lock-contention~\cite{Goetz:JCP06}).

% However, \code{Atomic} classes are not meant as a general replacement for
% locking. One can use them only when the critical sections for an object require
% updates to a \emph{single} variable. Sometimes an object needs to maintain an
% invariant involving more than one variable (e.g., a \code{ValueRange} object
% needs to ensure that its \code{max} int field is always greater than its
% \code{min} int field). In such cases the programmer cannot use an
% \code{AtomicInteger} to maintain a \emph{multi-variable} invariant, but
% needs to use a lock around code which updates those variables together.
% 
% Additionally, atomic classes are not a general purpose replacement for their
% related classes. \code{AtomicInteger} is \emph{mutable} while its equivalent
% \code{Integer} class is \emph{immutable}. Therefore, an \code{AtomicInteger}
% does not define \code{hashCode} and \code{equals} methods.  


% With all the restrictions on \code{Atomic} classes, why would a programmer use
% them? Using \code{Atomic} classes, the programmer gets both \emph{thread-safety}
% (built-in the \code{Atomic} classes) and \emph{scalability} (the lock-free
% updates reduce lock-contention under heavy accesses~\cite{Goetz:JCP06}).
 
\subsection{Code Transformations}


\tool enables a programmer to convert an \code{int} field to an
\code{AtomicInteger} field. \tool changes the declaration type of the
\code{int} field to \code{AtomicInteger} and replaces all field updates with 
their equivalent atomic API methods in \code{AtomicInteger}.

Figure~\ref{fig:CounterExample} shows how \tool refactors some code
from Apache Tomcat. We use this example to illustrate various 
transformations.

\myParagraph{Initialization}
Because the refactored \code{value} field is an \code{AtomicInteger} object,
\tool initializes it in the field initializer (otherwise a
\code{NullPointerException} is thrown the first time when a method is invoked
on \code{value}). \tool initializes the field with the
expression in the original field initializer (if present), or with the
default value \code{0}.


\myParagraph{Read/Write Accesses}
\tool replaces read access (e.g., in method \code{getCounter()}) with a call to
\code{AtomicInteger}'s \code{get()}. The \code{get()} method returns 
the \code{int} value encapsulated by the \code{AtomicInteger}. Thus a client of the \code{Counter}
would never get the \code{AtomicInteger} object, but only the \code{int}
value. This ensures that the client of the class cannot mutate the
primitive \code{int} field, a contract similar with the one in the sequential
version.

\tool replaces write accesses (e.g., in method \code{setCounter}) with a call
to \code{AtomicInteger}'s \code{set()}. 

\myParagraph{Expressions}
There are three kinds of update expressions: infix (e.g., \code{f = f + 2}),
prefix (e.g., \code{++f}), and postfix (e.g., \code{f++}). 
\tool rewrites an infix update expression using a call to the atomic
\code{addAndGet(int delta)}.
% There is no equivalent method or form for \code{*} or \code{/}, so they cannot
% be refactored. 
\tool rewrites a prefix update expression with a call to the atomic
\code{incrementAndGet()} (e.g., method~\code{inc()} in
Fig.~\ref{fig:CounterExample}). It rewrites a postfix expression with a call to
the atomic \code{getAndIncrement()}. 
\Comment{The difference in name between the
methods \code{getAndIncrement()} and \code{incrementAndGet()} reflects the
difference in their return values. \code{getAndIncrement()} increments its
operand but returns the old value, whereas \code{incrementAndGet()}
first increments the operand and then returns the new value.}
\tool rewrites similarly other expressions (e.g., the decrement expressions).
Since \code{AtomicInteger} does not offer atomic APIs for all operands (e.g.,
multiplication), \tool warns a user when an update expression cannot be replaced
by the atomic APIs.
%So, if \code{f} is an \code{int}, then \code{f = f+2} becomes \code{f.addAndGet(2)} and \code{f = f-2} becomes \code{f.addAndGet(-2)}.

\myParagraph{Synchronization}
If the original code contains \code{synchronized} accesses to the
refactored field, \tool tries to remove the synchronization since this becomes
superfluous after the conversion to \code{AtomicInteger} (thread-safety is
built-in the \code{AtomicInteger}). \tool does not remove the synchronized block
(i) if the refactored code contains more than one call to the atomic APIs (
since a context switch can still occur between two consecutive calls to atomic
APIs) and (ii) if the synchronized block accesses other fields in
the class (since the \code{AtomicInteger} can maintain invariants only over one
single variable).

\Comment{Here we presented the most relevant transformations. \tool applies other
transformations (e.g., for different kinds of expression operators) similar in
spirit with the ones presented.}

%\begin{itemize}
%  \item - initialization of AtomicInteger
%  \item - read/write access
%  \item - infix
%  \item - prefix/postfix
%  \item - removal of synchronized
%\end{itemize}


%---------------------------------------------
\section{Convert to ConcurrentHashMap}
\begin{figure*}[t]
\begin{CodeOut}
\begin{tabular}{@{}l|l@{}}
\begin{minipage}[t]{\MaxWidth}
\begin{verbatim}
// before refactoring
HashMap<Presence, Container> directedPresences;

public void directedPresenceSent(Presence presence,
     				 	 ChannelHandler handler, String jid) {
 ...
 Container<ChannelHandler, String> container;
 ...
 
 container = directedPresences.get(presence);

 if (container == null) {
  container = new Container<ChannelHandler, String>();
  container.add(handler, jid);
  directedPresences.put(presence, container);
 }

 ...
}
\end{verbatim}
\end{minipage}
&
\begin{minipage}[t]{\MaxWidth}
\begin{verbatim}
// after refactoring
ConcurrentHashMap<String, Container> directedPresences;

public void directedPresenceSent(Presence presence,
              ChannelHandler handler, String jid) {
 ...
 Container<ChannelHandler, String> container;
 ...
 
 directedPresences.putIfAbsent(presence,
                               createContainer(handler, jid));
 
 ...
}

Container<ChannelHandler, String> createContainer
                  (ChannelHandler handler, String jid) {
 Container container = new Container<ChannelHandler, String>();
 container.add(handler, jid);
 return container;
}

\end{verbatim}
\end{minipage}
\end{tabular}
\end{CodeOut}
\caption{Example ConvertToHashMap refactoring from Struts using the
\texttt{putIfAbsent} pattern.}
\label{fig:putIfAbsent}
\end{figure*}

\subsection{ConcurrentHashMap in Java}
The \code{j.u.c.} package contains several concurrent collection classes. 
\code{ConcurrentHashMap} is the thread-safe implementation for \code{HashMap}.

Before \code{j.u.c.}, a programmer could create a thread-safe \code{HashMap}
by using a \emph{common} lock to protect all \code{map} accesses. This results
in poor concurrency when multiple threads contend for the lock.

\code{ConcurrentHashMap} uses a different locking strategy that offers much
better scalability. It uses a \emph{fine-grained} locking mechanism (called
\emph{lock-stripping}) that enables all readers to run concurrently, and allows
a limited number of writers to update the map concurrently. The implementation
uses a number of \code{N} locks (the default value is 16), each of them guarding
a part of the hash buckets. Assuming that the hash functions spreads well the
values, and that keys are accessed randomly, this reduces the contention for
any given lock by a factor of \code{N}. Replacing a synchronized \code{HashMap}
with \code{ConcurrentHashMap} offers dramatic scalability improvements~\cite{Goetz:JCP06}.

\code{ConcurrentHashMap} implements the \code{Map} interface, therefore it
includes the API methods offered by \code{HashMap}. In addition, it contains
three APIs (\code{putIfAbsent}, \code{replace}, and a conditional
\code{remove}). Each API supersedes several calls to \code{Map} operations,
but it executes atomically. For example, \code{putIfAbsent} atomically (i)
checks whether the map contains a given key, and (ii) puts the key-value pair if
it is absent.




\subsection{Code Transformations}

\tool converts a \code{HashMap} field to a \\
\code{ConcurrentHashMap} field.

\myParagraph{Initialization and Accesses.} \tool changes the declaration and the
initialization of the field. Because \code{HashMap} and
\code{ConcurrentHashMap} implement the same interface (\code{Map}),
initialization and map accesses remain largely the same. 

\myParagraph{Map Updates.} \tool detects update code patterns and
replaces them with the appropriate \\
\code{ConcurrentHashMap} API method.

The patterns have a similar structure: (i) check whether the map contains a
certain key, and (ii) as appropriate, invoke one of \code{put}, \code{replace},
or \code{remove}). This structure can have small variations. 
For instance, the check can invoke \code{containsKey}, \code{get}, or
an equality check using \code{get}. A temporary variable might hold the result
of the check (like in Fig.~\ref{fig:putIfAbsent}). \tool handles all combinations among these map
update variations. Although these are the most common variations we have seen in
real code, there might be other variations that we have not seen yet. Failing
to convert those updates does not break user code if it contains a lock; it only
misses the oportunity to use the atomic update APIs.

Fig.~\ref{fig:putIfAbsent} shows an example where
the then-branch has two statements that create a new \code{container} value and
then places it in the map. \tool extracts these creational statements in a
\code{createContainer} method and calls this method when inserting the new value
in the \code{directedPresences} map. In addition, \tool performs a data-flow
analysis to find out whether the input value to the map
(e.g., \code{container}) is wrote in the conditional code, and read after the
conditional code. Accordingly, \tool saves the created value in a temporary
variable, and assigns it to the input variable in case that 
\code{putIfAbsent} succeeded (not shown in the figure).


%\begin{itemize}
%  \item initialization of ConcurrentHashMap
%  \item putIfAbsent (i) - no createValue,\\ 
%  (ii) with extracted method createValue
%  \item - replace
%  \item - remove
%  \item - description of variations with (i) containsKey, (ii) get
%\end{itemize}






% \begin{figure*}[t]
% \begin{CodeOut}
% \begin{tabular}{@{}l|l@{}}
% \begin{minipage}[t]{\MaxWidth}
% \begin{verbatim}
% // before refactoring
% private Map _registeredModules = new HashMap();
% 
% public void resetModuleConfig(String namespace) {
%  ...
%  ModuleConfig mc = (ModuleConfig)
%                 _registeredModules.get(namespace);
%  
%  if (mc != null) {
%   URL moduleConfURL = getModuleConfURL(namespace);
%   mc = new ModuleConfig(namespace, moduleConfURL);
%   
%   _registeredModules.put(namespace, mc);
%  }
% }
% \end{verbatim}
% \end{minipage}
% &
% \begin{minipage}[t]{\MaxWidth}
% \begin{verbatim}
% // after refactoring
% private Map _registeredModules = new ConcurrentHashMap();
% 
% public void resetModuleConfig(String namespace) {
%  ...
% 
%  _registeredModules.replace(namespace, createValue(namespace));
% }
% 
% ModuleConfig createValue(String namespace) {
%  
%  URL moduleConfURL = getModuleConfURL(namespace);
%  mc = new ModuleConfig(namespace, moduleConfURL);
%  
%  return mc;
% }
% \end{verbatim}
% \end{minipage}
% \end{tabular}
% \end{CodeOut}
% \caption{Example ConvertToHashMap refactoring of code adapted from Struts using the replace pattern, with get, an intermediate variable, and method extraction.}
% \label{fig:replace}
% \end{figure*}


%---------------------------------------------
\section{Preliminary Evaluation}

\myParagraph{Research Question.} To evaluate the effectiveness of \tool, we
answer the question ``How does manually-refactored code compare with code
refactored with \tool in terms of using the correct APIs and identifying all
oportunities to replace field accesses with thread-safe API calls?''

\myParagraph{Setup.} We start from code that was manually refactored by
open-source developers to use \code{AtomicInteger} or
\code{ConcurrentHashMap}. We then refactor \emph{the same} fields by
running \tool on the initial code. We compare the code refactored with \tool
against code refactored by hand. We look at places where the two outputs
differ, and quantify the number of \emph{errors} (i.e., one of the outputs
misuses the concurrent APIs) and the number of \emph{omissions} (i.e., the
refactored output could have used a concurrent API, but it instead uses the
obsolete, lock-protected APIs).

To get codebases which use \code{AtomicInteger} or \\
\code{ConcurrentHashMap} we
used 6 open-source projects: Apcahe Tomcat, MINA, Struts, GlassFish,
JaxLib, and Zimbra. We used the head versions available in the version control
system as of June 1st, 2008. 

\Comment{
Although for \code{AtomicInteger} we were able to
find both the version with the \code{int} field and the version with \code{AtomicInteger} field, for the second refactoring
(\code{ConvertToConcurrentHashMap}) we were not able to find the versions which 
contained \code{HashMap}. It seems that those projects were using
\code{ConcurrentHashMap} from the first version of the file. In those cases we
manually replaced \emph{only} the type declaration of the
\code{ConcurrentHashMap} field with \code{HashMap}; then we ran \tool to convert
the usage of \code{HashMap} back to \code{ConcurrentHashMap}
}

\myParagraph{Results.}
\tool applied all the correct transformations that the project developers
applied. We noticed several cases where \tool outperforms the developers: \tool
produces the correct code, or it identifies more opportunities for using the
new, scalable APIs.


For \emph{ConvertToAtomicInteger}, we noticed cases where
the developers used the wrong APIs when they refactored by hand. We noticed
that developers erroneously replaced infix expressions like \code{++f} with
\code{f.getAndIncrement()}, which is the equivalent API for the postfix expression. They should
have replaced \code{++f} with \code{f.incrementAndGet()}. The erroneous usage
of the API can cause an ``off-by-one'' value if the field is read in the same
statement that increments it. Table~\ref{tab:ImproperAIUsages} shows the number
of such human errors in the two projects.

 
For \emph{ConvertToConcurrentHashMap} we noticed cases when the developers
omitted to use the new atomic \code{putIfAbsent} and conditional \code{remove}
operations, and instead use the old patterns involving synchronized,
lock-protected access to \code{put} and \code{remove}. Although the
refactored code is thread-safe, this defeats the purpose of using
\code{ConcurrentHashMap}, i.e., to avoid locking the whole map for the 
duration of update operations in order to improve the scalability of the
application. Table~\ref{tab:ImproperCHMUsages} shows the number of such
human omissions in the studied projects.

%\begin{itemize}
%  \item
%  Describe the goal as a question: How does the code manually converted
%  compares with the code converted with \tool? More specifically, are there
%  cases when \tool outperforms the manual conversion?
%  Further explain what this means
%  \item These are the projects that we looked at (project version x.y.z)
%  \item setup of the experiment 
%  \item results:
%  how many times the human was wrong: misusing getAndIncrement vs incrementAndGet
%  explain this
%  MENTION TABLE
%  places where they missed to use putIfAbsent, replace, remove
%  (put a table here). 
%  \end{itemize}

\begin{table}
% increase table row spacing, adjust to taste
%\renewcommand{\arraystretch}{1.3}
\centering
\begin{footnotesize}
\begin{tabular}{l|cc|cc} 
         & \multicolumn{2}{|c|}{\textbf{incrementAndGet}} & \multicolumn{2}{|c}{\textbf{decrementAndGet}} \\ 
         & correct        & erroneous                     & correct        & erroneous                     \\ 
         & usages         &  usages                       & usages         & usages                        \\ \hline
  Tomcat &      0         &        1                      &        0       &     1                         \\
  MINA   &      0         &        1                      &        0       &     1                         \\
\end{tabular}
\caption{Human errors in using \texttt{AtomicInteger} updates in the 6
refactorings performed by the developers of Tomcat and MINA.}
\label{tab:ImproperAIUsages}
\end{footnotesize}
\vspace{-0.2in}
\end{table}

\begin{table}
% increase table row spacing, adjust to taste
\renewcommand{\arraystretch}{1.3}
\centering
\begin{footnotesize}
\begin{tabular}{l |cc|cc}

             & \multicolumn{2}{|c|}{\textbf{putIfAbsent}}  &\multicolumn{2}{|c}{\textbf{remove}} \\
             & correct& omissions & correct  & omissions  \\
             & usages &           & usages   &            \\ \hline
\ Struts     &  0     &  9        &  0       &  0  \\
\ GlassFish  &  0     &  8        &  1       &  3  \\
\ JaXLib     &  9     &  2        &  0       &  0  \\
\ Zimbra     &  3     &  42       &  11      &  6  \\
\end{tabular}
\caption{Human omissions to use \texttt{ConcurrentHashMap}'s
\texttt{putIfAbsent} and conditional \texttt{remove} in the 69
\texttt{ConcurrentHashMap} fields used by the developers of Struts,
GlassFish, JaxLib, and Zimbra.}
\label{tab:ImproperCHMUsages}
\end{footnotesize}
\vspace{-0.1 in}
\end{table}


%---------------------------------------------
\section{Related Work}

Balaban et al.~\cite{Balaban'05} present a tool for converting between obsolete
classes and their modern replacements. The programmer specifies a mapping
between the old APIs and the new APIs, and the tool uses a type-constraint
analysis to determine whether it can replace all usages of the obsolete class.
Their tool is more general than ours, since it can work with any API mapping,
for example one between \code{HashMap} and \code{ConcurrentHashMap}. \tool is
less general, since the conversion between \code{HashMap} and
\code{ConcurrentHashMap} is custom implemented. However, such conversion
requires more powerful AST pattern matching and rewriting than the one used in
their tool. Their tool can replace only a single PAI call at a time, whereas
our tool replaces a set of related but dispersed API calls (like the ones in
Fig.~\ref{fig:putIfAbsent}).

Boshernitsan et al.~\cite{Boshernitsan'07:FrameworkForTransformations} present
iXj, a general framework for code transformations. iXj has an intuitive user
interface that enables the user to quickly sketch a pattern for the code
transformation. Although useful for a broad range of transformations, iXj is not
able to transform code where the pattern matching executes against several
dispersed statements (like the ones in Fig.~\ref{fig:putIfAbsent}) and it
requires data-flow analysis. In such scenarios, a user needs to use a custom
implemented transformation tool like \tool.

%---------------------------------------------
\section{Conclusions}
Refactoring sequential code to parallelism is not a trivial task. Even
seemingly simple refactorings like replacing data types with thread-safe,
scalable implementations provided in \code{java.util.concurrent}, is prone to
human errors and omissions. In this paper we present \tool which automates two
refactorings for converting integer fields to \code{AtomicInteger} and hash maps to
\code{ConcurrentHashMap}. Our preliminary experience with \tool shows that it
is more effective than a human developer in identifying and applying such
transformations.

The refactorings presented here are ``enabling transformations'', i.e., they
make a program thread-safe, but do not introduce multi-threading into a
single-threaded program. Currently we are working on other transformations that
introduce multi-threading. One refactoring converts a sequential
recursive divide-and-conquer algorithm into one which runs the recursive
branches in parallel using \code{ForkJoinTask}s. Another refactoring
parallelizes a sequential loop by using the \code{ParallelArray} construct.


% \acks
% 
% We thank Adam Kiezun for feedback on a draft of this paper.

% \bibliographystyle{plainnat}
% 
% \begin{thebibliography}{10}
% @book{Goetz:JCP06,
%  author = {Brian Goetz and Tim Peierls and Joshua Bloch and Joseph Bowbeer and  David Holmes and Doug Lea},
%  title = {Java Concurrency in Practice},
%  year = {2006},
%  publisher = {Addison-Wesley},
%  }
% \end{thebibliography}
% \bibliography{../../../Bibliography/biblio}

\bibliographystyle{abbrv}
%\bibliography{./biblio}
\begin{thebibliography}{10}
\bibitem{Balaban'05}
I.~Balaban, F.~Tip, and R.~Fuhrer.
\newblock Refactoring support for class library migration.
\newblock In {\em OOPSLA '05}, pages 265--279. 

\bibitem{Boshernitsan'07:FrameworkForTransformations}
M.~Boshernitsan, S.~L. Graham, and M.~A. Hearst.
\newblock Aligning development tools with the way programmers think about code
  changes.
\newblock In {\em CHI '07}, pages 567--576. 

\bibitem{Dig'08:studyOfConcurrentTransformations}
D.~Dig, J.~Marrero, and M.~D. Ernst.
\newblock How do programs become more concurrent? {A} story of program
  transformations.
\newblock Technical Report MIT-CSAIL-TR-2008-053, MIT, September 2008.

\bibitem{Goetz:JCP06}
B.~Goetz, T.~Peierls, J.~Bloch, J.~Bowbeer, D.~Holmes, and D.~Lea.
\newblock {\em Java Concurrency in Practice}.
\newblock Addison-Wesley, 2006.

\bibitem{Lea:CPJ99}
D.~Lea.
\newblock {\em Concurrent Programming in Java. Second Edition: Design
  Principles and Patterns}.
\newblock Addison-Wesley, 1999.
\end{thebibliography}

\end{document}

